% Options for packages loaded elsewhere
\PassOptionsToPackage{unicode}{hyperref}
\PassOptionsToPackage{hyphens}{url}
%
\documentclass[
  ignorenonframetext,
]{beamer}
\usepackage{pgfpages}
\setbeamertemplate{caption}[numbered]
\setbeamertemplate{caption label separator}{: }
\setbeamercolor{caption name}{fg=normal text.fg}
\beamertemplatenavigationsymbolsempty
% Prevent slide breaks in the middle of a paragraph
\widowpenalties 1 10000
\raggedbottom
\setbeamertemplate{part page}{
  \centering
  \begin{beamercolorbox}[sep=16pt,center]{part title}
    \usebeamerfont{part title}\insertpart\par
  \end{beamercolorbox}
}
\setbeamertemplate{section page}{
  \centering
  \begin{beamercolorbox}[sep=12pt,center]{part title}
    \usebeamerfont{section title}\insertsection\par
  \end{beamercolorbox}
}
\setbeamertemplate{subsection page}{
  \centering
  \begin{beamercolorbox}[sep=8pt,center]{part title}
    \usebeamerfont{subsection title}\insertsubsection\par
  \end{beamercolorbox}
}
\AtBeginPart{
  \frame{\partpage}
}
\AtBeginSection{
  \ifbibliography
  \else
    \frame{\sectionpage}
  \fi
}
\AtBeginSubsection{
  \frame{\subsectionpage}
}
\usepackage{amsmath,amssymb}
\usepackage{iftex}
\ifPDFTeX
  \usepackage[T1]{fontenc}
  \usepackage[utf8]{inputenc}
  \usepackage{textcomp} % provide euro and other symbols
\else % if luatex or xetex
  \usepackage{unicode-math} % this also loads fontspec
  \defaultfontfeatures{Scale=MatchLowercase}
  \defaultfontfeatures[\rmfamily]{Ligatures=TeX,Scale=1}
\fi
\usepackage{lmodern}
\usetheme[]{Warsaw}
\ifPDFTeX\else
  % xetex/luatex font selection
\fi
% Use upquote if available, for straight quotes in verbatim environments
\IfFileExists{upquote.sty}{\usepackage{upquote}}{}
\IfFileExists{microtype.sty}{% use microtype if available
  \usepackage[]{microtype}
  \UseMicrotypeSet[protrusion]{basicmath} % disable protrusion for tt fonts
}{}
\makeatletter
\@ifundefined{KOMAClassName}{% if non-KOMA class
  \IfFileExists{parskip.sty}{%
    \usepackage{parskip}
  }{% else
    \setlength{\parindent}{0pt}
    \setlength{\parskip}{6pt plus 2pt minus 1pt}}
}{% if KOMA class
  \KOMAoptions{parskip=half}}
\makeatother
\usepackage{xcolor}
\newif\ifbibliography
\usepackage{color}
\usepackage{fancyvrb}
\newcommand{\VerbBar}{|}
\newcommand{\VERB}{\Verb[commandchars=\\\{\}]}
\DefineVerbatimEnvironment{Highlighting}{Verbatim}{commandchars=\\\{\}}
% Add ',fontsize=\small' for more characters per line
\usepackage{framed}
\definecolor{shadecolor}{RGB}{248,248,248}
\newenvironment{Shaded}{\begin{snugshade}}{\end{snugshade}}
\newcommand{\AlertTok}[1]{\textcolor[rgb]{0.94,0.16,0.16}{#1}}
\newcommand{\AnnotationTok}[1]{\textcolor[rgb]{0.56,0.35,0.01}{\textbf{\textit{#1}}}}
\newcommand{\AttributeTok}[1]{\textcolor[rgb]{0.13,0.29,0.53}{#1}}
\newcommand{\BaseNTok}[1]{\textcolor[rgb]{0.00,0.00,0.81}{#1}}
\newcommand{\BuiltInTok}[1]{#1}
\newcommand{\CharTok}[1]{\textcolor[rgb]{0.31,0.60,0.02}{#1}}
\newcommand{\CommentTok}[1]{\textcolor[rgb]{0.56,0.35,0.01}{\textit{#1}}}
\newcommand{\CommentVarTok}[1]{\textcolor[rgb]{0.56,0.35,0.01}{\textbf{\textit{#1}}}}
\newcommand{\ConstantTok}[1]{\textcolor[rgb]{0.56,0.35,0.01}{#1}}
\newcommand{\ControlFlowTok}[1]{\textcolor[rgb]{0.13,0.29,0.53}{\textbf{#1}}}
\newcommand{\DataTypeTok}[1]{\textcolor[rgb]{0.13,0.29,0.53}{#1}}
\newcommand{\DecValTok}[1]{\textcolor[rgb]{0.00,0.00,0.81}{#1}}
\newcommand{\DocumentationTok}[1]{\textcolor[rgb]{0.56,0.35,0.01}{\textbf{\textit{#1}}}}
\newcommand{\ErrorTok}[1]{\textcolor[rgb]{0.64,0.00,0.00}{\textbf{#1}}}
\newcommand{\ExtensionTok}[1]{#1}
\newcommand{\FloatTok}[1]{\textcolor[rgb]{0.00,0.00,0.81}{#1}}
\newcommand{\FunctionTok}[1]{\textcolor[rgb]{0.13,0.29,0.53}{\textbf{#1}}}
\newcommand{\ImportTok}[1]{#1}
\newcommand{\InformationTok}[1]{\textcolor[rgb]{0.56,0.35,0.01}{\textbf{\textit{#1}}}}
\newcommand{\KeywordTok}[1]{\textcolor[rgb]{0.13,0.29,0.53}{\textbf{#1}}}
\newcommand{\NormalTok}[1]{#1}
\newcommand{\OperatorTok}[1]{\textcolor[rgb]{0.81,0.36,0.00}{\textbf{#1}}}
\newcommand{\OtherTok}[1]{\textcolor[rgb]{0.56,0.35,0.01}{#1}}
\newcommand{\PreprocessorTok}[1]{\textcolor[rgb]{0.56,0.35,0.01}{\textit{#1}}}
\newcommand{\RegionMarkerTok}[1]{#1}
\newcommand{\SpecialCharTok}[1]{\textcolor[rgb]{0.81,0.36,0.00}{\textbf{#1}}}
\newcommand{\SpecialStringTok}[1]{\textcolor[rgb]{0.31,0.60,0.02}{#1}}
\newcommand{\StringTok}[1]{\textcolor[rgb]{0.31,0.60,0.02}{#1}}
\newcommand{\VariableTok}[1]{\textcolor[rgb]{0.00,0.00,0.00}{#1}}
\newcommand{\VerbatimStringTok}[1]{\textcolor[rgb]{0.31,0.60,0.02}{#1}}
\newcommand{\WarningTok}[1]{\textcolor[rgb]{0.56,0.35,0.01}{\textbf{\textit{#1}}}}
\setlength{\emergencystretch}{3em} % prevent overfull lines
\providecommand{\tightlist}{%
  \setlength{\itemsep}{0pt}\setlength{\parskip}{0pt}}
\setcounter{secnumdepth}{-\maxdimen} % remove section numbering
\ifLuaTeX
  \usepackage{selnolig}  % disable illegal ligatures
\fi
\usepackage{bookmark}
\IfFileExists{xurl.sty}{\usepackage{xurl}}{} % add URL line breaks if available
\urlstyle{same}
\hypersetup{
  pdftitle={Statistics with R, class 15},
  hidelinks,
  pdfcreator={LaTeX via pandoc}}

\title{Statistics with R, class 15}
\subtitle{Interval estimation, confidence intervals, \emph{t}
distribution}
\author{Grzegorz Krajewski\\
\strut \\
Faculty of Psychology\\
University of Warsaw\\
\strut \\
\href{mailto:krajewski@psych.uw.edu.pl}{\nolinkurl{krajewski@psych.uw.edu.pl}}}
\date{5 December 2024}

\begin{document}
\frame{\titlepage}

\section{Sampling distributions and
estimation}\label{sampling-distributions-and-estimation}

\begin{frame}{CLT and sampling distribution of mean}
\phantomsection\label{clt-and-sampling-distribution-of-mean}
\begin{center}\includegraphics[height=0.7\textheight]{class_15_confidenceIntervals_files/figure-beamer/clt-1} \end{center}

\begin{itemize}
\tightlist
\item
  \(\bar{X} \sim N\left(\mu, \frac{\sigma^2}{N}\right)\)
\item
  What about non-normal population distributions?
\end{itemize}
\end{frame}

\begin{frame}[fragile]{CLT and sampling distribution of mean}
\phantomsection\label{clt-and-sampling-distribution-of-mean-1}
\begin{center}\includegraphics[height=0.6\textheight]{class_15_confidenceIntervals_files/figure-beamer/clt2-1} \end{center}

\begin{itemize}
\tightlist
\item
  Conventional 5\%
\item
  What are the cut-off points for IQ given sample size \(N = 25\)?
\end{itemize}

\begin{Shaded}
\begin{Highlighting}[]
\SpecialCharTok{\textgreater{}} \FunctionTok{qnorm}\NormalTok{(}\FunctionTok{c}\NormalTok{(.}\DecValTok{025}\NormalTok{, .}\DecValTok{975}\NormalTok{), }\DecValTok{100}\NormalTok{, }\DecValTok{15}\SpecialCharTok{/}\FunctionTok{sqrt}\NormalTok{(}\DecValTok{25}\NormalTok{))}
\end{Highlighting}
\end{Shaded}
\end{frame}

\begin{frame}[fragile]{Standardised sampling distribution of mean}
\phantomsection\label{standardised-sampling-distribution-of-mean}
\begin{center}\includegraphics[height=0.7\textheight]{class_15_confidenceIntervals_files/figure-beamer/ssdm-1} \end{center}

\begin{itemize}
\tightlist
\item
  Standardised cut-off values around \texttt{zero} (same abs. value)
\end{itemize}
\end{frame}

\begin{frame}{Standardised sampling distribution of mean}
\phantomsection\label{standardised-sampling-distribution-of-mean-1}
\begin{center}\includegraphics[height=0.7\textheight]{class_15_confidenceIntervals_files/figure-beamer/ssdm2-1} \end{center}

\begin{itemize}
\tightlist
\item
  Easily applicable to any population parameters\ldots{}
\item
  \(\bar{X}_{2.5\%} = \mu \pm Z_{2.5\%} * \frac{\sigma}{\sqrt{N}}\)
\item
  \ldots{} and other cut-off points (e.g., \(Z_{0.5\%}\))
\end{itemize}
\end{frame}

\section{Confidence intervals}\label{confidence-intervals}

\begin{frame}{Confidence interval for population mean}
\phantomsection\label{confidence-interval-for-population-mean}
\begin{center}\includegraphics[height=0.7\textheight]{class_15_confidenceIntervals_files/figure-beamer/ci_plot-1} \end{center}

\(CI_{p\%} = \bar{X} \pm Z_{p\%} * \frac{\sigma}{\sqrt{N}}\)
\end{frame}

\begin{frame}[fragile]{Interval estimation}
\phantomsection\label{interval-estimation}
\begin{itemize}
\item
  \emph{p\% confidence interval}: We have \emph{p} \% confidence that
  the sample comes from a population with the parameter falling within
  the interval.
\item
  \(CI_{p\%} = \bar{X} \pm Z_{p\%} * \frac{\sigma}{\sqrt{N}}\)

  \begin{itemize}
  \tightlist
  \item
    \(\bar{X}\): sample mean
  \item
    \(N\): sample size
  \item
    \(\sigma\): population standard deviation
  \item
    \(Z_{p\%}\) value related to \emph{confidence level} p
    (\texttt{qnorm(p/2)})
  \end{itemize}
\end{itemize}
\end{frame}

\begin{frame}{Exercises on CI:}
\phantomsection\label{exercises-on-ci}
\begin{itemize}
\item
  30 CogSci students were sampled at random and their mean IQ is 117.
  Assuming that standard deviation in the population of CogSci students
  is the same as in general population, construct 99\% CI around the
  mean. How does it relate to the general population mean IQ? What does
  it tell us?
\item
  20 Psychology students were sampled at random and their mean IQ is
  105. Assuming that standard deviation in the population of Psychology
  students is the same as in general population, can we say with 95\%
  confidence level that mean IQ of Psychology students differs from the
  general population mean IQ?
\end{itemize}
\end{frame}

\begin{frame}[fragile]{Solutions (CI):}
\phantomsection\label{solutions-ci}
\begin{Shaded}
\begin{Highlighting}[]
\SpecialCharTok{\textgreater{}} \CommentTok{\#Ex.1}
\ErrorTok{\textgreater{}} \DecValTok{117} \SpecialCharTok{+} \FunctionTok{qnorm}\NormalTok{(.}\DecValTok{995}\NormalTok{) }\SpecialCharTok{*} \DecValTok{15} \SpecialCharTok{/} \FunctionTok{sqrt}\NormalTok{(}\DecValTok{30}\NormalTok{)}
\SpecialCharTok{\textgreater{}} \DecValTok{117} \SpecialCharTok{{-}} \FunctionTok{qnorm}\NormalTok{(.}\DecValTok{995}\NormalTok{) }\SpecialCharTok{*} \DecValTok{15} \SpecialCharTok{/} \FunctionTok{sqrt}\NormalTok{(}\DecValTok{30}\NormalTok{)}
\SpecialCharTok{\textgreater{}} \CommentTok{\# General population mean outside CI}
\ErrorTok{\textgreater{}} \CommentTok{\# CogSci population different}
\ErrorTok{\textgreater{}} 
\ErrorTok{\textgreater{}} \CommentTok{\#Ex.2}
\ErrorTok{\textgreater{}} \DecValTok{105} \SpecialCharTok{+} \FunctionTok{qnorm}\NormalTok{(.}\DecValTok{975}\NormalTok{) }\SpecialCharTok{*} \DecValTok{15} \SpecialCharTok{/} \FunctionTok{sqrt}\NormalTok{(}\DecValTok{20}\NormalTok{)}
\SpecialCharTok{\textgreater{}} \DecValTok{105} \SpecialCharTok{{-}} \FunctionTok{qnorm}\NormalTok{(.}\DecValTok{975}\NormalTok{) }\SpecialCharTok{*} \DecValTok{15} \SpecialCharTok{/} \FunctionTok{sqrt}\NormalTok{(}\DecValTok{20}\NormalTok{)}
\SpecialCharTok{\textgreater{}} \CommentTok{\# General population mean within CI}
\ErrorTok{\textgreater{}} \CommentTok{\# Psych population not different??}
\end{Highlighting}
\end{Shaded}
\end{frame}

\begin{frame}[fragile]{Simulation of CI}
\phantomsection\label{simulation-of-ci}
\begin{Shaded}
\begin{Highlighting}[]
\SpecialCharTok{\textgreater{}}\NormalTok{ i }\OtherTok{\textless{}{-}} \DecValTok{10000}
\SpecialCharTok{\textgreater{}}\NormalTok{ mu }\OtherTok{\textless{}{-}} \DecValTok{100}\NormalTok{; sigma }\OtherTok{\textless{}{-}} \DecValTok{15}\NormalTok{; n }\OtherTok{\textless{}{-}} \DecValTok{50}
\SpecialCharTok{\textgreater{}} \FunctionTok{mean}\NormalTok{(}\FunctionTok{replicate}\NormalTok{(i, \{}
\SpecialCharTok{+}\NormalTok{    s }\OtherTok{\textless{}{-}} \FunctionTok{rnorm}\NormalTok{(n, mu, sigma)}
\SpecialCharTok{+}\NormalTok{    mu }\SpecialCharTok{\textgreater{}=} \FunctionTok{mean}\NormalTok{(s) }\SpecialCharTok{{-}} \FunctionTok{qnorm}\NormalTok{(.}\DecValTok{975}\NormalTok{) }\SpecialCharTok{*}\NormalTok{ sigma}\SpecialCharTok{/}\FunctionTok{sqrt}\NormalTok{(n) }\SpecialCharTok{\&}
\SpecialCharTok{+}\NormalTok{    mu }\SpecialCharTok{\textless{}=} \FunctionTok{mean}\NormalTok{(s) }\SpecialCharTok{+} \FunctionTok{qnorm}\NormalTok{(.}\DecValTok{975}\NormalTok{) }\SpecialCharTok{*}\NormalTok{ sigma}\SpecialCharTok{/}\FunctionTok{sqrt}\NormalTok{(n)}
\SpecialCharTok{+}\NormalTok{ \}))}
\end{Highlighting}
\end{Shaded}

\begin{itemize}
\tightlist
\item
  Analyse the code and predict its outcome
\item
  Copy the code and run it
\item
  How do the values of N and sigma affect CIs?
\end{itemize}
\end{frame}

\begin{frame}{Simulation of CI}
\phantomsection\label{simulation-of-ci-1}
\begin{center}\includegraphics[height=0.6\textheight]{class_15_confidenceIntervals_files/figure-beamer/ci_plot1-1} \end{center}

\begin{itemize}
\tightlist
\item
  How do the values of N and sigma affect CIs?

  \begin{itemize}
  \tightlist
  \item
    How does CIs' width depend on them?
  \item
    What about the frequency with which the population mean falls within
    the CI?
  \item
    They affect the width of CI, never how often population mean falls
    within CI
  \end{itemize}
\end{itemize}
\end{frame}

\section{\texorpdfstring{\emph{t}
distribution}{t distribution}}\label{t-distribution}

\begin{frame}{Unknown population variance}
\phantomsection\label{unknown-population-variance}
\begin{itemize}
\tightlist
\item
  So far we were a bit unrealistic assuming we know population sd

  \begin{itemize}
  \tightlist
  \item
    in most cases, if we did we would know the mean as well
  \item
    What can we do about it??
  \end{itemize}
\item
  Use sample standard deviation as an estimate (could it possibly work?)
\item
  Modify the simulation accordingly and check the outcome
\end{itemize}
\end{frame}

\begin{frame}[fragile]{Unknown population variance}
\phantomsection\label{unknown-population-variance-1}
\begin{Shaded}
\begin{Highlighting}[]
\SpecialCharTok{\textgreater{}}\NormalTok{ i }\OtherTok{\textless{}{-}} \DecValTok{10000}
\SpecialCharTok{\textgreater{}}\NormalTok{ mu }\OtherTok{\textless{}{-}} \DecValTok{165}\NormalTok{; sigma }\OtherTok{\textless{}{-}} \DecValTok{5}\NormalTok{; N }\OtherTok{\textless{}{-}} \DecValTok{20}
\SpecialCharTok{\textgreater{}} \FunctionTok{mean}\NormalTok{(}\FunctionTok{replicate}\NormalTok{(i, \{}
\SpecialCharTok{+}\NormalTok{    s }\OtherTok{\textless{}{-}} \FunctionTok{rnorm}\NormalTok{(N, mu, sigma)}
\SpecialCharTok{+}\NormalTok{    mu }\SpecialCharTok{\textgreater{}=} \FunctionTok{mean}\NormalTok{(s) }\SpecialCharTok{{-}} \FunctionTok{qnorm}\NormalTok{(.}\DecValTok{975}\NormalTok{) }\SpecialCharTok{*} \FunctionTok{sd}\NormalTok{(s)}\SpecialCharTok{/}\FunctionTok{sqrt}\NormalTok{(N) }\SpecialCharTok{\&}
\SpecialCharTok{+}\NormalTok{    mu }\SpecialCharTok{\textless{}=} \FunctionTok{mean}\NormalTok{(s) }\SpecialCharTok{+} \FunctionTok{qnorm}\NormalTok{(.}\DecValTok{975}\NormalTok{) }\SpecialCharTok{*} \FunctionTok{sd}\NormalTok{(s)}\SpecialCharTok{/}\FunctionTok{sqrt}\NormalTok{(N)}
\SpecialCharTok{+}\NormalTok{ \}))}
\end{Highlighting}
\end{Shaded}

\begin{itemize}
\tightlist
\item
  How does N affect CIs?

  \begin{itemize}
  \tightlist
  \item
    Check the outcome for \(N=10\) and \(N=100\)
  \item
    The actual confidence level lower than expected 95\%
  \item
    The greater N the smaller the difference
  \end{itemize}
\item
  Why is the CI width systematically underestimated?
\end{itemize}
\end{frame}

\begin{frame}{\emph{t} distribution}
\phantomsection\label{t-distribution-1}
\begin{center}\includegraphics[height=0.8\textheight]{class_15_confidenceIntervals_files/figure-beamer/td-1} \end{center}
\end{frame}

\begin{frame}{\emph{t} distribution}
\phantomsection\label{t-distribution-2}
\begin{center}\includegraphics[height=0.8\textheight]{class_15_confidenceIntervals_files/figure-beamer/td_tail-1} \end{center}
\end{frame}

\begin{frame}[fragile]{\emph{t} distribution}
\phantomsection\label{t-distribution-3}
\begin{itemize}
\tightlist
\item
  Modify the simulation accordingly

  \begin{itemize}
  \tightlist
  \item
    using \texttt{qt()}
  \end{itemize}
\end{itemize}

\begin{Shaded}
\begin{Highlighting}[]
\SpecialCharTok{\textgreater{}}\NormalTok{ i }\OtherTok{\textless{}{-}} \DecValTok{10000}
\SpecialCharTok{\textgreater{}}\NormalTok{ mu }\OtherTok{\textless{}{-}} \DecValTok{165}\NormalTok{; sigma }\OtherTok{\textless{}{-}} \DecValTok{5}\NormalTok{; N }\OtherTok{\textless{}{-}} \DecValTok{20}
\SpecialCharTok{\textgreater{}} \FunctionTok{mean}\NormalTok{(}\FunctionTok{replicate}\NormalTok{(i, \{}
\SpecialCharTok{+}\NormalTok{    s }\OtherTok{\textless{}{-}} \FunctionTok{rnorm}\NormalTok{(N, mu, sigma)}
\SpecialCharTok{+}\NormalTok{    mu }\SpecialCharTok{\textgreater{}=} \FunctionTok{mean}\NormalTok{(s) }\SpecialCharTok{{-}} \FunctionTok{qt}\NormalTok{(.}\DecValTok{975}\NormalTok{, N}\DecValTok{{-}1}\NormalTok{) }\SpecialCharTok{*} \FunctionTok{sd}\NormalTok{(s)}\SpecialCharTok{/}\FunctionTok{sqrt}\NormalTok{(N) }\SpecialCharTok{\&}
\SpecialCharTok{+}\NormalTok{    mu }\SpecialCharTok{\textless{}=} \FunctionTok{mean}\NormalTok{(s) }\SpecialCharTok{+} \FunctionTok{qt}\NormalTok{(.}\DecValTok{975}\NormalTok{, N}\DecValTok{{-}1}\NormalTok{) }\SpecialCharTok{*} \FunctionTok{sd}\NormalTok{(s)}\SpecialCharTok{/}\FunctionTok{sqrt}\NormalTok{(N)}
\SpecialCharTok{+}\NormalTok{ \}))}
\end{Highlighting}
\end{Shaded}
\end{frame}

\begin{frame}[fragile]{\emph{t} distribution and CIs -- exercise}
\phantomsection\label{t-distribution-and-cis-exercise}
\begin{itemize}
\item
  \(CI_{p\%} = \bar{X} \pm t_{p\%}(df) * \frac{s}{\sqrt{N}}\)
\item
  25 people were randomly sampled, their mean shoe size is 40 and the
  standard deviation of their shoe sizes is 3.
\item
  Using \emph{t} distribution, calculate the upper limit of 95\% CI
  around mean.
\end{itemize}

\begin{Shaded}
\begin{Highlighting}[]
\SpecialCharTok{\textgreater{}} \DecValTok{40} \SpecialCharTok{+} \FunctionTok{qt}\NormalTok{(.}\DecValTok{975}\NormalTok{, }\DecValTok{24}\NormalTok{) }\SpecialCharTok{*} \DecValTok{3} \SpecialCharTok{/} \FunctionTok{sqrt}\NormalTok{(}\DecValTok{25}\NormalTok{)}
\end{Highlighting}
\end{Shaded}
\end{frame}

\end{document}
