% Options for packages loaded elsewhere
\PassOptionsToPackage{unicode}{hyperref}
\PassOptionsToPackage{hyphens}{url}
%
\documentclass[
  ignorenonframetext,
]{beamer}
\usepackage{pgfpages}
\setbeamertemplate{caption}[numbered]
\setbeamertemplate{caption label separator}{: }
\setbeamercolor{caption name}{fg=normal text.fg}
\beamertemplatenavigationsymbolsempty
% Prevent slide breaks in the middle of a paragraph
\widowpenalties 1 10000
\raggedbottom
\setbeamertemplate{part page}{
  \centering
  \begin{beamercolorbox}[sep=16pt,center]{part title}
    \usebeamerfont{part title}\insertpart\par
  \end{beamercolorbox}
}
\setbeamertemplate{section page}{
  \centering
  \begin{beamercolorbox}[sep=12pt,center]{part title}
    \usebeamerfont{section title}\insertsection\par
  \end{beamercolorbox}
}
\setbeamertemplate{subsection page}{
  \centering
  \begin{beamercolorbox}[sep=8pt,center]{part title}
    \usebeamerfont{subsection title}\insertsubsection\par
  \end{beamercolorbox}
}
\AtBeginPart{
  \frame{\partpage}
}
\AtBeginSection{
  \ifbibliography
  \else
    \frame{\sectionpage}
  \fi
}
\AtBeginSubsection{
  \frame{\subsectionpage}
}
\usepackage{amsmath,amssymb}
\usepackage{iftex}
\ifPDFTeX
  \usepackage[T1]{fontenc}
  \usepackage[utf8]{inputenc}
  \usepackage{textcomp} % provide euro and other symbols
\else % if luatex or xetex
  \usepackage{unicode-math} % this also loads fontspec
  \defaultfontfeatures{Scale=MatchLowercase}
  \defaultfontfeatures[\rmfamily]{Ligatures=TeX,Scale=1}
\fi
\usepackage{lmodern}
\usetheme[]{Warsaw}
\ifPDFTeX\else
  % xetex/luatex font selection
\fi
% Use upquote if available, for straight quotes in verbatim environments
\IfFileExists{upquote.sty}{\usepackage{upquote}}{}
\IfFileExists{microtype.sty}{% use microtype if available
  \usepackage[]{microtype}
  \UseMicrotypeSet[protrusion]{basicmath} % disable protrusion for tt fonts
}{}
\makeatletter
\@ifundefined{KOMAClassName}{% if non-KOMA class
  \IfFileExists{parskip.sty}{%
    \usepackage{parskip}
  }{% else
    \setlength{\parindent}{0pt}
    \setlength{\parskip}{6pt plus 2pt minus 1pt}}
}{% if KOMA class
  \KOMAoptions{parskip=half}}
\makeatother
\usepackage{xcolor}
\newif\ifbibliography
\usepackage{color}
\usepackage{fancyvrb}
\newcommand{\VerbBar}{|}
\newcommand{\VERB}{\Verb[commandchars=\\\{\}]}
\DefineVerbatimEnvironment{Highlighting}{Verbatim}{commandchars=\\\{\}}
% Add ',fontsize=\small' for more characters per line
\usepackage{framed}
\definecolor{shadecolor}{RGB}{248,248,248}
\newenvironment{Shaded}{\begin{snugshade}}{\end{snugshade}}
\newcommand{\AlertTok}[1]{\textcolor[rgb]{0.94,0.16,0.16}{#1}}
\newcommand{\AnnotationTok}[1]{\textcolor[rgb]{0.56,0.35,0.01}{\textbf{\textit{#1}}}}
\newcommand{\AttributeTok}[1]{\textcolor[rgb]{0.13,0.29,0.53}{#1}}
\newcommand{\BaseNTok}[1]{\textcolor[rgb]{0.00,0.00,0.81}{#1}}
\newcommand{\BuiltInTok}[1]{#1}
\newcommand{\CharTok}[1]{\textcolor[rgb]{0.31,0.60,0.02}{#1}}
\newcommand{\CommentTok}[1]{\textcolor[rgb]{0.56,0.35,0.01}{\textit{#1}}}
\newcommand{\CommentVarTok}[1]{\textcolor[rgb]{0.56,0.35,0.01}{\textbf{\textit{#1}}}}
\newcommand{\ConstantTok}[1]{\textcolor[rgb]{0.56,0.35,0.01}{#1}}
\newcommand{\ControlFlowTok}[1]{\textcolor[rgb]{0.13,0.29,0.53}{\textbf{#1}}}
\newcommand{\DataTypeTok}[1]{\textcolor[rgb]{0.13,0.29,0.53}{#1}}
\newcommand{\DecValTok}[1]{\textcolor[rgb]{0.00,0.00,0.81}{#1}}
\newcommand{\DocumentationTok}[1]{\textcolor[rgb]{0.56,0.35,0.01}{\textbf{\textit{#1}}}}
\newcommand{\ErrorTok}[1]{\textcolor[rgb]{0.64,0.00,0.00}{\textbf{#1}}}
\newcommand{\ExtensionTok}[1]{#1}
\newcommand{\FloatTok}[1]{\textcolor[rgb]{0.00,0.00,0.81}{#1}}
\newcommand{\FunctionTok}[1]{\textcolor[rgb]{0.13,0.29,0.53}{\textbf{#1}}}
\newcommand{\ImportTok}[1]{#1}
\newcommand{\InformationTok}[1]{\textcolor[rgb]{0.56,0.35,0.01}{\textbf{\textit{#1}}}}
\newcommand{\KeywordTok}[1]{\textcolor[rgb]{0.13,0.29,0.53}{\textbf{#1}}}
\newcommand{\NormalTok}[1]{#1}
\newcommand{\OperatorTok}[1]{\textcolor[rgb]{0.81,0.36,0.00}{\textbf{#1}}}
\newcommand{\OtherTok}[1]{\textcolor[rgb]{0.56,0.35,0.01}{#1}}
\newcommand{\PreprocessorTok}[1]{\textcolor[rgb]{0.56,0.35,0.01}{\textit{#1}}}
\newcommand{\RegionMarkerTok}[1]{#1}
\newcommand{\SpecialCharTok}[1]{\textcolor[rgb]{0.81,0.36,0.00}{\textbf{#1}}}
\newcommand{\SpecialStringTok}[1]{\textcolor[rgb]{0.31,0.60,0.02}{#1}}
\newcommand{\StringTok}[1]{\textcolor[rgb]{0.31,0.60,0.02}{#1}}
\newcommand{\VariableTok}[1]{\textcolor[rgb]{0.00,0.00,0.00}{#1}}
\newcommand{\VerbatimStringTok}[1]{\textcolor[rgb]{0.31,0.60,0.02}{#1}}
\newcommand{\WarningTok}[1]{\textcolor[rgb]{0.56,0.35,0.01}{\textbf{\textit{#1}}}}
\setlength{\emergencystretch}{3em} % prevent overfull lines
\providecommand{\tightlist}{%
  \setlength{\itemsep}{0pt}\setlength{\parskip}{0pt}}
\setcounter{secnumdepth}{-\maxdimen} % remove section numbering
\AtBeginSubsection{}
\ifLuaTeX
  \usepackage{selnolig}  % disable illegal ligatures
\fi
\usepackage{bookmark}
\IfFileExists{xurl.sty}{\usepackage{xurl}}{} % add URL line breaks if available
\urlstyle{same}
\hypersetup{
  pdftitle={Statistics with R, class 17},
  hidelinks,
  pdfcreator={LaTeX via pandoc}}

\title{Statistics with R, class 17}
\subtitle{\emph{t} tests}
\author{Grzegorz Krajewski\\
\strut \\
Faculty of Psychology\\
University of Warsaw\\
\strut \\
\href{mailto:krajewski@psych.uw.edu.pl}{\nolinkurl{krajewski@psych.uw.edu.pl}}}
\date{12 December 2024}

\begin{document}
\frame{\titlepage}

\section{\texorpdfstring{Previously in ``Statistics with
\emph{R}''\ldots{}}{Previously in ``Statistics with R''\ldots{}}}\label{previously-in-statistics-with-r}

\begin{frame}{Standardised sampling distribution of mean}
\phantomsection\label{standardised-sampling-distribution-of-mean}
\begin{center}\includegraphics[height=0.7\textheight]{class_17_t_tests_files/figure-beamer/ssdm-1} \end{center}
\end{frame}

\begin{frame}{Standardised sampling distribution of mean}
\phantomsection\label{standardised-sampling-distribution-of-mean-1}
\begin{center}\includegraphics[height=0.7\textheight]{class_17_t_tests_files/figure-beamer/ssdm2-1} \end{center}
\end{frame}

\begin{frame}{\emph{t} distribution}
\phantomsection\label{t-distribution}
\begin{center}\includegraphics[height=0.8\textheight]{class_17_t_tests_files/figure-beamer/td-1} \end{center}
\end{frame}

\begin{frame}{\emph{t} distribution}
\phantomsection\label{t-distribution-1}
\begin{center}\includegraphics[height=0.8\textheight]{class_17_t_tests_files/figure-beamer/td_tail-1} \end{center}
\end{frame}

\begin{frame}{Confidence interval around sample mean}
\phantomsection\label{confidence-interval-around-sample-mean}
\begin{center}\includegraphics[height=0.8\textheight]{class_17_t_tests_files/figure-beamer/ci_plot-1} \end{center}
\end{frame}

\begin{frame}{Standardised sampling distribution of mean}
\phantomsection\label{standardised-sampling-distribution-of-mean-2}
\begin{center}\includegraphics[height=0.7\textheight]{class_17_t_tests_files/figure-beamer/ssdm_0-1} \end{center}
\end{frame}

\begin{frame}{Standardised sampling distribution of mean}
\phantomsection\label{standardised-sampling-distribution-of-mean-3}
\begin{center}\includegraphics[height=0.7\textheight]{class_17_t_tests_files/figure-beamer/ssdm2_0-1} \end{center}
\end{frame}

\section{One-sample tests of mean}\label{one-sample-tests-of-mean}

\begin{frame}{Lady tasting tea}
\phantomsection\label{lady-tasting-tea}
\begin{center}\includegraphics[height=0.8\textheight]{class_17_t_tests_files/figure-beamer/binom-1} \end{center}
\end{frame}

\begin{frame}{Statistical tests}
\phantomsection\label{statistical-tests}
\begin{itemize}
\tightlist
\item
  Assume \emph{null hypothesis}, \(h_0\) (perhaps make other necessary
  \emph{assumptions} too).
\item
  Decide on \emph{significance level} \(\alpha\) (conventionally default
  to 0.05).
\item
  Construct \emph{sample distribution} of the \emph{test statistic},
  assuming the truth of \(h_0\) (and considering the other assumptions).
\item
  Identify \emph{critical region} (based on \(\alpha\)).
\item
  Draw a random sample, collect measurements, calculate the statistic.
\item
  Reject, or not, \(h_0\).
\end{itemize}
\end{frame}

\begin{frame}{\emph{z} test}
\phantomsection\label{z-test}
\begin{itemize}
\tightlist
\item
  30 CogSci students' mean IQ is 105.
\item
  Do they come from the population with mean IQ = 100?
\item
  \(h_0\): \(\mu_0 = 100\)
\item
  \(z = (\bar{X} - \mu_0) / (\sigma / \sqrt{N})\)
\end{itemize}

\begin{center}\includegraphics[height=0.6\textheight]{class_17_t_tests_files/figure-beamer/z_test-1} \end{center}
\end{frame}

\begin{frame}[fragile]{One sample \emph{t} test (1)}
\phantomsection\label{one-sample-t-test-1}
\begin{itemize}
\tightlist
\item
  From many years' experience we know that CogSci students pass a
  particular exam with an average score of 26.5. This year there are 10
  new students and their scores are:
\end{itemize}

\begin{Shaded}
\begin{Highlighting}[]
\SpecialCharTok{\textgreater{}}\NormalTok{ scores }\OtherTok{\textless{}{-}} \FunctionTok{c}\NormalTok{(}\DecValTok{10}\NormalTok{, }\DecValTok{50}\NormalTok{, }\DecValTok{46}\NormalTok{, }\DecValTok{32}\NormalTok{, }\DecValTok{37}\NormalTok{, }\DecValTok{28}\NormalTok{, }\DecValTok{41}\NormalTok{, }\DecValTok{20}\NormalTok{, }\DecValTok{32}\NormalTok{, }\DecValTok{43}\NormalTok{)}
\end{Highlighting}
\end{Shaded}

\begin{itemize}
\tightlist
\item
  Are these students better than an average CogSci student?

  \begin{itemize}
  \tightlist
  \item
    I.e., do they come from a population with a different mean?
  \end{itemize}
\item
  \(h_0\): \(\mu_0 = 26.5\)
\item
  \(t = (\bar{X} - \mu_0) / (s / \sqrt{N})\)
\end{itemize}
\end{frame}

\begin{frame}[fragile]{One sample \emph{t} test (2)}
\phantomsection\label{one-sample-t-test-2}
\begin{center}\includegraphics[height=0.6\textheight]{class_17_t_tests_files/figure-beamer/t_test_1-1} \end{center}

\begin{Shaded}
\begin{Highlighting}[]
\SpecialCharTok{\textgreater{}} \FunctionTok{qt}\NormalTok{(.}\DecValTok{975}\NormalTok{, }\FunctionTok{length}\NormalTok{(scores) }\SpecialCharTok{{-}} \DecValTok{1}\NormalTok{)}
\SpecialCharTok{\textgreater{}}\NormalTok{ t.statistic }\OtherTok{\textless{}{-}}\NormalTok{ (}\FunctionTok{mean}\NormalTok{(scores) }\SpecialCharTok{{-}} \FloatTok{26.5}\NormalTok{) }\SpecialCharTok{/}
\SpecialCharTok{+}\NormalTok{      (}\FunctionTok{sd}\NormalTok{(scores) }\SpecialCharTok{/} \FunctionTok{sqrt}\NormalTok{(}\FunctionTok{length}\NormalTok{(scores)))}
\end{Highlighting}
\end{Shaded}
\end{frame}

\begin{frame}[fragile]{One sample \emph{t} test (3)}
\phantomsection\label{one-sample-t-test-3}
\begin{center}\includegraphics[height=0.6\textheight]{class_17_t_tests_files/figure-beamer/t_test_2-1} \end{center}

\begin{Shaded}
\begin{Highlighting}[]
\SpecialCharTok{\textgreater{}} \FunctionTok{pt}\NormalTok{(t.statistic, }\FunctionTok{length}\NormalTok{(scores) }\SpecialCharTok{{-}} \DecValTok{1}\NormalTok{, }\AttributeTok{lower.tail=}\NormalTok{F) }\SpecialCharTok{*} \DecValTok{2}
\SpecialCharTok{\textgreater{}} \CommentTok{\# Times 2 to take into account the other tail!}
\end{Highlighting}
\end{Shaded}
\end{frame}

\section{\texorpdfstring{Paired-sample \emph{t}
test}{Paired-sample t test}}\label{paired-sample-t-test}

\begin{frame}[fragile]{Paired-sample \emph{t} test (1)}
\phantomsection\label{paired-sample-t-test-1}
\begin{itemize}
\tightlist
\item
  Two paired samples

  \begin{itemize}
  \tightlist
  \item
    e.g., repeated measures, siblings etc.
  \item
    differences between paired measurements
  \item
    \(h_0\): mean difference (\(M\)) = 0
  \end{itemize}
\item
  49 CogSci students took IQ test before and after first semester and
  for each of them the difference between both scores was calculated
\item
  The mean difference was 0.5 (in favour of the second measurement)
\item
  Standard deviation of the differences was 1.75. Can we say that a
  semester of studying CogSci improves intelligence?
\item
  \(t = M / (s / \sqrt{N})\)
\end{itemize}

\begin{Shaded}
\begin{Highlighting}[]
\SpecialCharTok{\textgreater{}}\NormalTok{ .}\DecValTok{5} \SpecialCharTok{/}\NormalTok{ (}\FloatTok{1.75} \SpecialCharTok{/} \FunctionTok{sqrt}\NormalTok{(}\DecValTok{49}\NormalTok{))}
\SpecialCharTok{\textgreater{}} \FunctionTok{pt}\NormalTok{(}\DecValTok{2}\NormalTok{, }\DecValTok{48}\NormalTok{, }\AttributeTok{lower.tail=}\NormalTok{F) }\SpecialCharTok{*} \DecValTok{2}
\end{Highlighting}
\end{Shaded}
\end{frame}

\begin{frame}[fragile]{Paired-sample \emph{t} test (2)}
\phantomsection\label{paired-sample-t-test-2}
\begin{itemize}
\tightlist
\item
  American scientists compared sexual satisfaction of women and men
  staying in long term relationships
\item
  15 couples were chosen at random and each person estimated their
  satisfaction on 1-5 scale
\item
  There are 15 answers from men and 15 answers from women. Why should we
  compare them pairwise?
\end{itemize}

\begin{Shaded}
\begin{Highlighting}[]
\SpecialCharTok{\textgreater{}}\NormalTok{ m }\OtherTok{\textless{}{-}} \FunctionTok{c}\NormalTok{(}\DecValTok{1}\NormalTok{, }\DecValTok{1}\NormalTok{, }\DecValTok{2}\NormalTok{, }\DecValTok{2}\NormalTok{, }\DecValTok{3}\NormalTok{, }\DecValTok{3}\NormalTok{, }\DecValTok{3}\NormalTok{, }\DecValTok{4}\NormalTok{, }\DecValTok{4}\NormalTok{, }\DecValTok{4}\NormalTok{, }\DecValTok{4}\NormalTok{, }\DecValTok{5}\NormalTok{, }\DecValTok{5}\NormalTok{, }\DecValTok{5}\NormalTok{, }\DecValTok{5}\NormalTok{)}
\SpecialCharTok{\textgreater{}}\NormalTok{ f }\OtherTok{\textless{}{-}} \FunctionTok{c}\NormalTok{(}\DecValTok{1}\NormalTok{, }\DecValTok{1}\NormalTok{, }\DecValTok{1}\NormalTok{, }\DecValTok{2}\NormalTok{, }\DecValTok{2}\NormalTok{, }\DecValTok{2}\NormalTok{, }\DecValTok{3}\NormalTok{, }\DecValTok{3}\NormalTok{, }\DecValTok{3}\NormalTok{, }\DecValTok{3}\NormalTok{, }\DecValTok{4}\NormalTok{, }\DecValTok{4}\NormalTok{, }\DecValTok{4}\NormalTok{, }\DecValTok{5}\NormalTok{, }\DecValTok{5}\NormalTok{)}
\SpecialCharTok{\textgreater{}}\NormalTok{ m }\SpecialCharTok{{-}}\NormalTok{ f }\CommentTok{\# differences we want to analyse}
\end{Highlighting}
\end{Shaded}

\begin{Shaded}
\begin{Highlighting}[]
\SpecialCharTok{\textgreater{}} \FunctionTok{mean}\NormalTok{(m}\SpecialCharTok{{-}}\NormalTok{f) }\SpecialCharTok{/}\NormalTok{ (}\FunctionTok{sd}\NormalTok{(m}\SpecialCharTok{{-}}\NormalTok{f) }\SpecialCharTok{/} \FunctionTok{sqrt}\NormalTok{(}\FunctionTok{length}\NormalTok{(m}\SpecialCharTok{{-}}\NormalTok{f)))}
\end{Highlighting}
\end{Shaded}
\end{frame}

\section{\texorpdfstring{Two Sample \emph{t}
test}{Two Sample t test}}\label{two-sample-t-test}

\begin{frame}{Two independent samples}
\phantomsection\label{two-independent-samples}
\begin{itemize}
\tightlist
\item
  Most of the time we deal with two samples that are not matched
\item
  \(h_0\): both samples come from populations with the same mean
  (\(\mu_1 - \mu_2 = 0\))
\item
  Statistic: difference between sample means (\(\bar{X_1} - \bar{X_2}\))
\item
  How does sampling distribution of \(\bar{X_1} - \bar{X_2}\) look like?
\item
  \emph{Additional assumption}: both samples come from populations with
  the same variance (\(\sigma^2_1 - \sigma^2_2 = 0\))
\end{itemize}
\end{frame}

\begin{frame}[fragile]{Sampling distribution of
\(\bar{X_1} - \bar{X_2}\) (1)}
\phantomsection\label{sampling-distribution-of-barx_1---barx_2-1}
\begin{itemize}
\tightlist
\item
  We know sampling distribution of a single mean

  \begin{itemize}
  \tightlist
  \item
    and can simulate it\ldots{}
  \end{itemize}
\end{itemize}

\begin{Shaded}
\begin{Highlighting}[]
\SpecialCharTok{\textgreater{}}\NormalTok{ N }\OtherTok{\textless{}{-}} \DecValTok{25}\NormalTok{; mu }\OtherTok{\textless{}{-}} \DecValTok{100}\NormalTok{; sigma }\OtherTok{\textless{}{-}} \DecValTok{15}
\SpecialCharTok{\textgreater{}}\NormalTok{ i }\OtherTok{\textless{}{-}} \DecValTok{10000}
\SpecialCharTok{\textgreater{}}\NormalTok{ sampling\_mean }\OtherTok{\textless{}{-}} \FunctionTok{replicate}\NormalTok{(i,}\FunctionTok{mean}\NormalTok{(}\FunctionTok{rnorm}\NormalTok{(N,mu,sigma)))}
\end{Highlighting}
\end{Shaded}

\begin{itemize}
\tightlist
\item
  Check its shape, mean, and standard deviation (and variance)
\end{itemize}

\begin{Shaded}
\begin{Highlighting}[]
\SpecialCharTok{\textgreater{}} \FunctionTok{hist}\NormalTok{(sampling\_mean, }\AttributeTok{prob =}\NormalTok{ T) }\CommentTok{\# simulation}
\SpecialCharTok{\textgreater{}} \FunctionTok{curve}\NormalTok{(}\FunctionTok{dnorm}\NormalTok{(x, mu, sigma }\SpecialCharTok{/} \FunctionTok{sqrt}\NormalTok{(N)), }\AttributeTok{add=}\NormalTok{T) }\CommentTok{\# theory}
\end{Highlighting}
\end{Shaded}

\begin{Shaded}
\begin{Highlighting}[]
\SpecialCharTok{\textgreater{}} \FunctionTok{mean}\NormalTok{(sampling\_mean) }\CommentTok{\# equals mu (population mean)}
\end{Highlighting}
\end{Shaded}

\begin{Shaded}
\begin{Highlighting}[]
\SpecialCharTok{\textgreater{}} \FunctionTok{sd}\NormalTok{(sampling\_mean) }\CommentTok{\# simulation}
\SpecialCharTok{\textgreater{}}\NormalTok{ sigma }\SpecialCharTok{/} \FunctionTok{sqrt}\NormalTok{(N) }\CommentTok{\# theory}
\end{Highlighting}
\end{Shaded}
\end{frame}

\begin{frame}[fragile]{Sampling distribution of
\(\bar{X_1} - \bar{X_2}\) (2)}
\phantomsection\label{sampling-distribution-of-barx_1---barx_2-2}
\begin{itemize}
\tightlist
\item
  Simulate sampling distribution of \(\bar{X_1} - \bar{X_2}\)
\end{itemize}

\begin{Shaded}
\begin{Highlighting}[]
\SpecialCharTok{\textgreater{}}\NormalTok{ sampling\_diff }\OtherTok{\textless{}{-}} \FunctionTok{replicate}\NormalTok{(i,}
\SpecialCharTok{+}                            \FunctionTok{mean}\NormalTok{(}\FunctionTok{rnorm}\NormalTok{(N, mu, sigma)) }\SpecialCharTok{{-}}
\SpecialCharTok{+}                            \FunctionTok{mean}\NormalTok{(}\FunctionTok{rnorm}\NormalTok{(N, mu, sigma)))}
\SpecialCharTok{\textgreater{}} \FunctionTok{hist}\NormalTok{(sampling\_diff)}
\end{Highlighting}
\end{Shaded}

\begin{Shaded}
\begin{Highlighting}[]
\SpecialCharTok{\textgreater{}} \FunctionTok{mean}\NormalTok{(sampling\_diff) }\CommentTok{\# equals 0 (consistent with h0)}
\SpecialCharTok{\textgreater{}} \FunctionTok{var}\NormalTok{(sampling\_diff)}
\SpecialCharTok{\textgreater{}}\NormalTok{ (sigma}\SpecialCharTok{\^{}}\DecValTok{2} \SpecialCharTok{/}\NormalTok{ N) }\SpecialCharTok{*} \DecValTok{2}
\end{Highlighting}
\end{Shaded}
\end{frame}

\begin{frame}[fragile]{Sampling distribution of
\(\bar{X_1} - \bar{X_2}\) (3)}
\phantomsection\label{sampling-distribution-of-barx_1---barx_2-3}
\begin{itemize}
\tightlist
\item
  Change parameters of one of the populations and see what happens
\end{itemize}

\begin{Shaded}
\begin{Highlighting}[]
\SpecialCharTok{\textgreater{}}\NormalTok{ mu1 }\OtherTok{\textless{}{-}} \DecValTok{100}\NormalTok{; sigma1 }\OtherTok{\textless{}{-}} \DecValTok{15}
\SpecialCharTok{\textgreater{}}\NormalTok{ mu2 }\OtherTok{\textless{}{-}} \DecValTok{110}\NormalTok{; sigma2 }\OtherTok{\textless{}{-}} \DecValTok{10}
\SpecialCharTok{\textgreater{}}\NormalTok{ sampling\_diff }\OtherTok{\textless{}{-}} \FunctionTok{replicate}\NormalTok{(i,}
\SpecialCharTok{+}                            \FunctionTok{mean}\NormalTok{(}\FunctionTok{rnorm}\NormalTok{(N, mu1, sigma1)) }\SpecialCharTok{{-}}
\SpecialCharTok{+}                            \FunctionTok{mean}\NormalTok{(}\FunctionTok{rnorm}\NormalTok{(N, mu2, sigma2)))}
\end{Highlighting}
\end{Shaded}

\begin{Shaded}
\begin{Highlighting}[]
\SpecialCharTok{\textgreater{}} \FunctionTok{hist}\NormalTok{(sampling\_diff)}
\end{Highlighting}
\end{Shaded}

\begin{Shaded}
\begin{Highlighting}[]
\SpecialCharTok{\textgreater{}} \FunctionTok{mean}\NormalTok{(sampling\_diff) }\CommentTok{\# mu1 {-} mu2}
\SpecialCharTok{\textgreater{}} \FunctionTok{var}\NormalTok{(sampling\_diff)}
\SpecialCharTok{\textgreater{}}\NormalTok{ sigma1}\SpecialCharTok{\^{}}\DecValTok{2} \SpecialCharTok{/}\NormalTok{ N }\SpecialCharTok{+}\NormalTok{ sigma2}\SpecialCharTok{\^{}}\DecValTok{2} \SpecialCharTok{/}\NormalTok{ N }\CommentTok{\# sum of both variances}
\end{Highlighting}
\end{Shaded}
\end{frame}

\begin{frame}{Sampling distribution of \(\bar{X_1} - \bar{X_2}\) (4)}
\phantomsection\label{sampling-distribution-of-barx_1---barx_2-4}
\begin{itemize}
\tightlist
\item
  Sampling distribution of \(\bar{X_1} - \bar{X_2}\):

  \begin{itemize}
  \tightlist
  \item
    Normal
  \item
    \(\mu = \mu_1 - \mu_2\)
  \item
    \(\sigma_{\bar{X_1} - \bar{X_2}} = \sqrt{\sigma^2_{\bar{X_1}} + \sigma^2_{\bar{X_2}}}\)
  \end{itemize}
\end{itemize}
\end{frame}

\begin{frame}{Sampling distribution of \(\bar{X_1} - \bar{X_2}\) (5)}
\phantomsection\label{sampling-distribution-of-barx_1---barx_2-5}
\begin{center}\includegraphics[height=0.8\textheight]{class_17_t_tests_files/figure-beamer/sampling_diff_mean_plot-1} \end{center}
\end{frame}

\begin{frame}{Standardised sampling distribution of
\(\bar{X_1} - \bar{X_2}\)}
\phantomsection\label{standardised-sampling-distribution-of-barx_1---barx_2}
\begin{center}\includegraphics[height=0.8\textheight]{class_17_t_tests_files/figure-beamer/sampling_diff_mean_z_plot-1} \end{center}
\end{frame}

\begin{frame}[fragile]{Student's \emph{t} test}
\phantomsection\label{students-t-test}
\begin{itemize}
\tightlist
\item
  \(\sigma_1\) and \(\sigma_2\) are unknown
\item
  Assumption: \(\sigma_1 = \sigma_2\)
\item
  \(s_p\): weighted mean of \(s_1\) and \(s_2\)
\item
  \(t = \frac{\bar{X_1} - \bar{X_2}}{s_p * \sqrt{1/N_1 + 1/N_2}}\)
\item
  \(df = N_1 + N_2 - 2\)
\item
  If assumption \(\sigma_1 = \sigma_2\) does not hold, Welch's
  correction reduces \(df\)
\end{itemize}

\begin{Shaded}
\begin{Highlighting}[]
\SpecialCharTok{\textgreater{}} \FunctionTok{t.test}\NormalTok{(m, f, }\AttributeTok{paired =} \ConstantTok{TRUE}\NormalTok{)}
\SpecialCharTok{\textgreater{}} \FunctionTok{t.test}\NormalTok{(m, f)}
\SpecialCharTok{\textgreater{}} \FunctionTok{t.test}\NormalTok{(m, f, }\AttributeTok{var.equal =} \ConstantTok{TRUE}\NormalTok{)}
\end{Highlighting}
\end{Shaded}
\end{frame}

\begin{frame}{Dependent \emph{vs} independent \emph{t} test}
\phantomsection\label{dependent-vs-independent-t-test}
\begin{itemize}
\tightlist
\item
  dependent (paired samples) \emph{vs} independent \emph{t} test:

  \begin{itemize}
  \tightlist
  \item
    \(t = \bar{D}/\frac{s_D}{\sqrt{N_D}}\),\\
    where \emph{D} = difference within pair
  \item
    \(t = (\bar{X_1} - \bar{X_2})/(\sqrt{s_p/N_1 + s_p/N_2})\),\\
    where \(s_p\) pooled from \(s_1\) and \(s_2\)
  \item
    denominator: \emph{standard error}
  \end{itemize}
\item
  (not only) \emph{t} test: an \emph{effect} to \emph{noise} ratio

  \begin{itemize}
  \tightlist
  \item
    pairing measurements reduces random variation (noise)
  \end{itemize}
\end{itemize}
\end{frame}

\end{document}
